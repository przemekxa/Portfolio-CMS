\documentclass[polish,polish,a4paper]{article}
\usepackage[utf8]{inputenc}
\usepackage[T1]{fontenc}
\usepackage[polish]{babel}
\usepackage{anysize}
\usepackage{indentfirst}
\usepackage{makecell}
\usepackage{graphicx}
\usepackage{hyperref}
\hypersetup{
    colorlinks=true,
    linkcolor=blue,
    filecolor=magenta,
    urlcolor=blue,
    pdftitle={Overleaf Example},
    pdfpagemode=FullScreen,
}

\marginsize{1.8cm}{1.8cm}{2cm}{2cm}

\title{Laboratorium z Systemów zarządzania treścią}
\author{
    Błażej Celmer (141197)\\
    Przemysław Ambroży (141182)\\\\
    grupa L1\\
    1. sem., Informatyka (ZTI) II st., studia niestacjonarne, WIiT
}

\begin{document}
\maketitle

\newpage

\tableofcontents

\newpage

\section{Charakterystyka projektu}

TODO

\section{Wymagania}

TODO

\subsection{Aktorzy}

TODO

\subsection{Wymagania funkcjonalne}

TODO

\subsection{Wymagania pozafunkcjonalne}

TODO

\section{Architektura systemu}

TODO

\subsection{Narzędzia}

TODO

\section{Schemat bazy danych}

TODO

\section{Diagramy UML}

\subsection{Diagram przypadków użycia}

TODO

\subsection{Diagram klas}

TODO

\section{Projekt interfejsu graficznego}

TODO

\section{Kod aplikacji}

TODO

\section{Analiza bezpieczeństwa}

TODO

\section{Podsumowanie}

TODO

\subsection*{Podział prac}

Błażej Celmer

\begin{itemize}
    \item TODO
\end{itemize}

Przemysław Ambroży

\begin{itemize}
    \item TODO
\end{itemize}

\subsection*{Realizacja celów}

TODO

\subsection*{Perspektywa rozwoju}

TODO

\end{document}